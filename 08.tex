\chapter{考察}
\label{chap:kosatsu}

本章では、ハイパー楽譜システム利用者の意見や自身の運用経験をまとめ、諸問題や新しい可能性について述べる。

\newpage

\section{評価}
本システムのプロトタイプとなるChrome拡張機能「hyperscorebox」を2018年11月21日にChrome Web Storeで公開した\footnote{\textsf{https://chrome.google.com/webstore/detail/hyperscorebox/cjlhoobllhkpjjomlijlfdblgifcdmoh}}。2018年12月現在、60名程度にインストールされている。
また、Web技術や音楽アプリ関連情報を共有する以下の開発者イベントにて hyperscorebox の展示発表を行った。
\begin{itemize}
    \item WebAudio.tokyo \#6\footnote{\textsf{https://webaudiotokyo.connpass.com/event/103821/}} (2018年11月15日 株式会社ドリコム)
    \item HTML5 Conference 2018\footnote{\textsf{https://events.html5j.org/conference/2018/11/}} (2018年11月25日 電気通信大学)
    \item Web Music Developers Meetup \#3\footnote{\textsf{https://connpass.com/event/112300/}} (2018年12月21日 株式会社デジタルハーツ)
\end{itemize}
本節では、
\begin{itemize}
    \item 筆者の運用経験
    \item ユーザーからのフィードバック
    \item 各種展示発表でのフィードバック
\end{itemize}
をまとめる。

\subsection{筆者の運用経験}
主に楽器練習/音楽理論の勉強/採譜といった用途で3ヶ月に渡り利用した。
\paragraph*{楽器練習}
筆者はクラシックギターの練習に本システムを利用しており、練習中に残した様々なメモを整理して記録できている。
以前まではたくさん書き込んでも練習する曲それぞれの楽譜に散逸してしまうため整理や再利用が難しく、結局役に立たないという問題に悩まされていたが、本システムによって解決された。
\paragraph*{音楽理論の勉強}
既存の教材に強く苦手意識を抱いていたが、自身が演奏している楽曲と知識の関係性がハイパーリンクによって可視化され、自分専用の分かりやすい教材として利用できている。
また音声の埋め込みによって実際の演奏を聞きながら学習でき、イメージしづらい概念の理解を深めるのに特に有効である。
\paragraph*{採譜}
筆者は本システムの利用前までABCの利用経験が無かったが、複雑な楽譜でない限りは楽譜IMEの支援によって快適に楽譜編集できている。
また採譜した楽曲の音声/動画/作曲者/演奏者といった周辺情報を一元管理でき便利である。

\subsection{感想・意見}
まず、第\ref{mondai}節で述べた楽譜の問題点に対して多くの賛同を得られた。
多くのユーザーは楽譜に不便さを感じていたものの、根本的な解決手段を見つけられない状況に置かれている。
その他の感想や意見として主に以下のようなものが挙げられた。

\begin{enumerate}
    \item 楽譜やコメントを瞬時に共有できる\\
    楽譜や演奏に対する要望を伝えるには「3小節目の1拍裏のドの音を半音上げて」「100小節のシドレミのフレーズをもっと大きく」といった具体的な指示を行う必要がある。
    本システムでは楽譜と一緒にコメントを書いたり、その場で楽譜を更新して瞬時に共有できるので、コミュニケーションコストを大幅に小さくできるのではないかという好意的な評価を得た。
    \item 既存の楽譜作成システムを使うには大げさなシーンで価値がある\\
    既存の楽譜作成システムを使って1.のような楽譜共有を実現する場合、
    \begin{enumerate}
        \item ソフトを起動して楽譜入力する
        \item スクリーンショットを撮影する
        \item 画像をSNS等で送信する
    \end{enumerate}という作業が必要である。
    そもそも楽譜作成システムは断片的な楽譜の作成を前提としていないため、楽譜そのものとは関係ないレイアウトや設定を意識する必要があり、このような用途では大袈裟である。
    一方本システムでは素早く書けてすぐ共有できるため好評であった。
    \item Webリソースを活用できる\\
    紙やPDF上の楽譜は静的で独立したメディアなので、外部の情報を活用したりリンクすることができない。
    マルチメディア情報をはじめとするWeb上の様々なリソースを活用しながら楽譜編集するという応用は新鮮なものとして受け止められ、大きな可能性を感じられるという感想を得た。
\end{enumerate}

\subsection{問題点}
問題点として以下のようなものが挙げられた。
\begin{enumerate}
    \item リンク機能について\\
    複数の音符にハイパーリンクが設定されている場合、すべての音符が1つのリンクを示しているのか、それぞれの音符が別のリンクを示しているのか一目見て理解できないという意見が挙げられた。
    これはリンク先にかかわらず全て青色の音符として表示しているためであるが、単音に対しても、フレーズや小節に対してもリンクを設定したいので、何らかのインタフェース的工夫が必要である。
    \item 楽譜編集について\\
    既存の楽譜を取り込めるような仕組みがほしいという意見が挙げられた。
    楽譜を自ら編集することを前提としているため、既存の楽曲を利用したい場合は元になる楽譜を別途用意したり、採譜したりして自分で入力する必要がある。
    楽譜IMEによってある程度入力作業が簡単になるものの、すぐ楽譜を利用したいユーザーにとっては大きな負担になる。
    \item 利用環境について\\
    タブレット上でも快適に本システムを利用したいという意見が挙げられた。
    PCブラウザ上での利用を前提とした設計のため、タブレットのようなタッチインタフェース端末では利用しづらい。
    既存の楽譜ビューアーはタブレットで利用されるものがほとんどであり、譜面台の上のような限られたスペースでも快適に利用できるため好まれている。
\end{enumerate}

\section{考察}
\subsection{設計の妥当性}
本システムは既存のものとは全く異なる新しい楽譜の利用形態を持つが、実際に利用したりデモを体験したユーザーからは概ね好意的に受け入れられ、Wikiと楽譜記述言語を組み合わせるという本システムの設計指針は正しかったといえる。
また本研究で述べた楽譜に対する問題意識にも多くの共感を得られたことから、本システムをベースにして様々な改善や拡張を行うことで、より良い楽譜利用環境を生み出せると考えられる。
\subsection{解決すべき課題}
\begin{enumerate}
    \item 音符のリンク状況が分からない\\
    音符にリンクが設定されている場合、リンク先を把握するにはABCを確認するか、実際に音符をクリックする必要があり非効率である。
    Webブラウザではフォーカスされたリンクに下線を付けたり、画面左下にリンク先アドレスを表示することで解決している。
    本システムでも同様の手法が有効であると考えられる。
    \item 自分で楽譜を書く必要がある\\
    IMSLP\footnote{\textsf{https://imslp.org}}といったパブリックドメイン楽譜配布サイトや出版社による販売サイト上には膨大な数の楽譜が存在し、ユーザーは必要とする楽譜をWebから簡単に見つけ出すことができるため、通常自ら楽譜を書く必要はない。
    本システムでも必要な楽譜をすぐ利用できるようにするために、既存の楽譜ファイルからABCへコンバートできる機能や、ユーザーが楽譜を配布できるWiki上のプラットフォームが必要であると考えられる。
    \item タブレットで使いづらい\\
    楽譜利用環境として主流であるタブレット上でも快適に利用できるようにするために、タッチインターフェースに適した以下のようなインタフェースの工夫が必要であると考えられる。
    \begin{itemize}
        \item ボタン/ペン操作による楽譜入力
        \item ハンズフリーな楽譜スクロール
    \end{itemize}
\end{enumerate}

\subsection{楽譜の問題点の検証}
本システムにおいて第\ref{mondai}節で述べた問題点が克服されているかどうかを問題点毎に検証する。
\begin{itemize}
    \item 簡単に編集できない\\
    新規作成/既存楽譜の編集両方を、楽譜IMEの支援を受けながら簡単に行える。
    \item メモなどの情報を自在に書けない\\
    楽譜と同じ場所にテキスト/マルチメディア情報を自在に書ける。
    \item 参照や管理が面倒\\
    楽譜内のハイパーリンクによって関連する楽譜や情報を辿って参照できる。
\end{itemize}
以上のように、第\ref{mondai}節で述べたすべての点に関して問題が解決していることがわかる。