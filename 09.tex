\chapter{結論}
\label{chap:kekka}

本章では本研究を総括する。

\newpage

\section{研究の成果}
本研究では、Wikiと楽譜記述言語を組み合わせた新しい楽譜システム「ハイパー楽譜システム」と、日本語入力システム的アプローチによる楽譜入力支援システム「楽譜IME」の提案を行った。

まず第\ref{chap:haikei}章において、楽譜の問題点をテキストの進化と比較しながら分析した。楽譜を扱う既存システムの現状をとりあげ、計算機が普及した現在も根本的に解決されていないことを示した。

第\ref{chap:sekkei}章では、第\ref{chap:haikei}章で述べた楽譜の問題点に対する有効的な解決方法を提案した。また、それに基づき本研究で開発した「ハイパー楽譜システム」の基本構成と使い方について述べた。

第\ref{chap:jissou}章では、「ハイパー楽譜システム」のアプリケーション構成と詳細な実装について述べた。

第\ref{chap:ime}章では、テキストによる楽譜入力には限界があることを示し、日本語入力システム的アプローチによる楽譜入力支援システム「楽譜IME」を提案した。

第\ref{chap:ouyou}章では、「ハイパー楽譜システム」によって実現可能な応用例について述べた。

第\ref{chap:kanren}章では、本研究に関連する研究を紹介し、それぞれのアプローチの特徴と問題点を分析した。

第\ref{chap:kosatsu}章では、筆者による運用経験やユーザーからのフィードバックをもとに本研究の有効性と問題点を分析した。

\section{総括}
本研究では楽譜を簡単に書いたり音符など以外の情報を自在に書いたりハイパーリンクを利用できる「ハイパー楽譜システム」とABCの入力を支援する「楽譜IME」の開発を行った。
ハイパー楽譜システムはWikiと楽譜記述言語の組み合わせによって既存の楽譜の問題点を克服するだけでなく、新しい楽譜の使い方を実現した。
楽譜IMEは日本語入力システムのアプローチを取ることでテキストによる楽譜入力の弱点を克服し、既存の楽譜作成システムに劣らない快適な楽譜入力を可能にした。
今後は第\ref{chap:kosatsu}章で述べた問題点についての改善や、システムの拡張を行っていく。