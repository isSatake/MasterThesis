\chapter{序論}
\label{chap:introduction}

本章では本研究の動機と目的、および本論文の構成について述べる。

\newpage

\section{研究の動機}

楽譜は17世紀ごろに五線譜による近代記譜法のフォーマットが確立してからほとんど形態が変わっておらず、今日においても国際的かつ普遍的な記法として広く利用されている。
これは楽譜が優れた表現力を持ち、様々な音楽を記録するのに適していることにほかならない。
しかし当初から印刷を前提としたフォーマットであったため、以下のような問題も存在する。

\begin{itemize}
    \item 簡単に編集できない
    \item メモなどの情報を自在に書けない
    \item 参照や管理が面倒
\end{itemize}

また計算機が普及した現在では楽譜を快適に閲覧したり、美しい楽譜を作成できるソフトウェアも広く利用されているが、PDFのように内容の変更が不可能な形式で楽譜を管理することが一般的であり、紙の楽譜と本質的な特徴は変わっていない。
一方、楽譜と同様に紙の上に記録されていたテキストは計算機の登場によって以下のような進化を遂げた。

\begin{itemize}
    \item コピー/ペースト/Undo/Redoといったテキスト編集支援機能が手軽な編集を可能にした
    \item IMEといったテキスト入力支援機能が手軽な入力を可能にした
    \item ハイパーテキストがマルチメディアを活用した分かりやすいドキュメントの作成や、ハイパーリンクを可能にした
    \item Webが様々なドキュメントへの素早いアクセスを可能にした
    \item Wikiのようなコラボレーションツールが複数人による共同編集を可能にした
\end{itemize}

これらの進化によって、紙というメディアの特性上ドキュメントが物理的に独立していて内容の変更も難しいという性質が消えてしまった。
かつては楽譜と同じ問題を抱えていたが、計算機によって新しい便利な使い方が発明され、広く普及したのである。
楽譜はフォーマットや使い方が数百年変わっていないので、計算機によって進化する余地はまだまだ存在すると考えられる。

\section{研究の目的}
本研究では、上記のような楽譜が持つ問題点を解決し、これまでの楽譜の在り方にとらわれない次世代の楽譜システム「ハイパー楽譜システム」の構築を目的とする。
ハイパーテキストやWikiの手法を取り入れることで、従来の楽譜システムでは実現できなかった快適な楽譜編集・利用環境を実現する。

\section{本論文の構成}

本論文は以下の9章で構成される。

第2章では、本研究の背景をより詳細に分析し、楽譜の問題点を整理する。

第3章では、本論文で提案するシステムの基本構成と使い方について述べる。

第4章では、本論文で提案するシステムの詳細な実装について述べる。

第5章では、本論文で提案するシステムで利用できる楽譜IMEを提案する。

第6章では、本論文で提案するシステムによって実現可能な応用例について述べる。

第7章では、関連する研究を紹介し、それらの特徴や本研究との関連を述べる。

第8章では、筆者による運用経験やユーザーからのフィードバックをまとめ、本論文で提案するシステムの有効性と問題点について述べる。

最後に、第9章で本論文のまとめと結論を述べる。
